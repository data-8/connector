\documentclass[11pt]{article}
\usepackage{fancyhdr}
\usepackage[vmargin=3.5cm, hmargin=2cm]{geometry}
\usepackage{amsmath, amsfonts, amsthm}
\usepackage{graphicx}
\usepackage{moreverb}
\usepackage{enumerate}
\usepackage{bm}
\usepackage[tiny,compact]{titlesec}
\pagestyle{fancy}
\headheight 14pt
\lhead{UC Berkeley}
\chead{}
\rhead{Stat 88 Fall 2016}
\rfoot{}
\cfoot{\thepage}
\lfoot{}

\parindent0pt  % to stop indenting paragraphs
\parskip1.5ex  % to insert vertical space between paragraphs

% Different font in captions
\newcommand{\captionfonts}{\small}

\makeatletter  % Allow the use of @ in command names
\long\def\@makecaption#1#2{%
  \vskip\abovecaptionskip
  \sbox\@tempboxa{{\captionfonts #1: #2}}%
  \ifdim \wd\@tempboxa >\hsize
    {\captionfonts #1: #2\par}
  \else
    \hbox to\hsize{\hfil\box\@tempboxa\hfil}%
  \fi
  \vskip\belowcaptionskip}
\makeatother   % Cancel the effect of \makeatletter

\newcommand{\V}{\mathrm{Var}}
\newcommand{\E}{\mathbb{E}}
\newcommand{\mbf}{\mathbf}
\newcommand{\mr}{\mathrm}
\newcommand{\yiobs}{Y_i^\mr{obs}}
\newcommand{\yobs}{Y^\mr{obs}}
\newcommand{\yimis}{Y_i^\mr{mis}}
\newcommand{\ymis}{Y^\mr{mis}}
\newcommand{\N}{\mathcal{N}}
\newcommand{\utilde}{\underset{\widetilde{}}}
\newcommand{\txt}{\texttt}

\begin{document}
\centerline{\textbf{Homework 8}}
\centerline{Due on Gradescope 11/22/2016 at 4:00PM (Before Lecture)}

\begin{enumerate}
\item \underline{Lost order:} Let $\bm x$ be a list of three numbers $\{1, 3, 5\}$. Let $\bm y$ be a list of another three numbers $\{2, 4, 6\}$, but
suppose that you lost the order of these three numbers, so you're not sure whether $\bm y$ is $\{4, 2, 6\}$, or $\{6, 4, 2\}$, etc.
    \begin{enumerate}
        \item What is the arrangement of numbers in $\bm y$ that has maximum correlation with $\bm x$?
        \item What is the arrangement of numbers in $\bm y$ that has minimum correlation with $\bm x$?
    \end{enumerate}

\item \underline{Variance of a sum:}
We have two lists of numbers $\bm x$ and $\bm y$ each with $n$ elements. $\bm x$ has average $\bar x$ and standard deviation $s_x$;
$\bm y$ has average $\bar y$ and standard deviation $s_y$.
$\bm x$ and $\bm y$ have covariance $Cov(\bm x, \bm y)$.

 We construct a new list $\bm z$ with $n$ elements, whose entries are determined by $z_i = x_i + y_i$.

\begin{enumerate}
\item Write the standard deviation of $z$ in terms of $\bar x$, $\bar y$, $s_x$, $s_y$, and $Cov(\bm x, \bm y)$.
    (Hint: Use the computational forms of the variance and covariance).
\item How does correlation between the lists $\bm x$ and $\bm y$ affect the variance of the list $\bm z$?
\end{enumerate}

\item \underline{Freebie from polling notebook:}
In the polling example we discussed, we had two lists with length $N$ composed of 1's and 0's: $\bm c$ of candidate preferences for each person in the population, and $\bm t$ of voter turnout status.
We said that the vote share for the candidate of interest was given by:
$$
\mu_v = \frac{\sum_{i=1}^N t_i c_i}{\sum_{i=1}^N t_i},
$$
but because it is difficult to measure $t_i$ for each person, we wanted to figure out whether we could instead measure population candidate prefernece instead:
$$
\mu_c = \frac{1}{N} \sum_{i=1}^N c_i.
$$

Prove that if $t_i$ and $c_i$ have correlation 0, then $\mu_c = \mu_v$.
\end{enumerate}
\end{document}
