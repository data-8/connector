\documentclass[11pt]{article}
\usepackage{fancyhdr}
\usepackage[vmargin=3.5cm, hmargin=2cm]{geometry}
\usepackage{amsmath, amsfonts, amsthm}
\usepackage{graphicx}
\usepackage{moreverb}
\usepackage{enumerate}
\usepackage{bm}
\usepackage[tiny,compact]{titlesec}
\pagestyle{fancy}
\headheight 14pt
\lhead{UC Berkeley}
\chead{}
\rhead{Stat 88 Fall 2016}
\rfoot{}
\cfoot{\thepage}
\lfoot{}

\parindent0pt  % to stop indenting paragraphs
\parskip1.5ex  % to insert vertical space between paragraphs

% Different font in captions
\newcommand{\captionfonts}{\small}

\makeatletter  % Allow the use of @ in command names
\long\def\@makecaption#1#2{%
  \vskip\abovecaptionskip
  \sbox\@tempboxa{{\captionfonts #1: #2}}%
  \ifdim \wd\@tempboxa >\hsize
    {\captionfonts #1: #2\par}
  \else
    \hbox to\hsize{\hfil\box\@tempboxa\hfil}%
  \fi
  \vskip\belowcaptionskip}
\makeatother   % Cancel the effect of \makeatletter

\newcommand{\V}{\mathrm{Var}}
\newcommand{\E}{\mathbb{E}}
\newcommand{\mbf}{\mathbf}
\newcommand{\mr}{\mathrm}
\newcommand{\yiobs}{Y_i^\mr{obs}}
\newcommand{\yobs}{Y^\mr{obs}}
\newcommand{\yimis}{Y_i^\mr{mis}}
\newcommand{\ymis}{Y^\mr{mis}}
\newcommand{\N}{\mathcal{N}}
\newcommand{\utilde}{\underset{\widetilde{}}}
\newcommand{\txt}{\texttt}

\begin{document}
\centerline{\textbf{Homework 7}}
\centerline{Due on Gradescope 11/1/2016 at 4:00PM (Before Lecture)}

\begin{enumerate}
\item \underline{Bounds}:
    \begin{enumerate}
    \item A non-negative random variable $X$ has expectation $E(X) = 100$. Provide the best bound you can for the probability $P(X \geq 500)$.
    \item Let $X$ be a binomial random variable with size $100$ and probability $p$. Recall that $E(X) = 100p$ and $\mathrm{SD}(X) = 10\sqrt{p(1-p)}$. In terms of $p$, provide the best bound you can for the probability that $X$ is more than 10 away from its mean.
    \end{enumerate}

\item \underline{Rolling Dice}:
    You roll a die with 6 faces 10 times. What is the expected number of faces that appear twice? (Hint: Express ``the number of faces that appear twice'' as a sum of indicator random variables).

\item \underline{Shirts, from Stat 88 Spring 2016}: Hipsters run a factory that makes button-down shirts from locally sourced materials. 
The factory has $n$ workers and is open 365 days each year. 
Each day the factory is open, each worker makes 6 shirts. 
But here is the what makes this factory special. 
The management has a rule that they shut down the factory and everyone takes off work whenever any worker has a birthday. 
The factory is open on all other days. 

How many workers should the management hire to maximize the expected number of shirts produced in a year? Answer the question in the following steps. Assume that each worker's birthday is selected at random with replacement from 365 possible days.
    \begin{enumerate}
    \item In terms of $n$, what is the probability that the factory will be open on a specific day (say, November 1)?
    \item Let $X$ be the number of days that the factory is open in a year. What is $E(X)$, the expected number of days that the factory will be open?

    \item Let $Y$ be the number of shirts the factory makes in a year. What is $E(Y)$, the expected number of shirts the factory will produce?
    \item Use calculus to find how many workers the management should hire.
    \end{enumerate}

\item \underline{Variance and Dependence}: Let $X$ and $Y$ be random variables that can take values 0 or 1.  Under these three scenarios, find $P(X = x, Y = y)$ for all valid values of $x$ and $y$ (you can write this as a table), $E(X + Y)$, and $\V(X+Y)$. These scenarios should be familiar from lecture. As you do this question, note how the variance of the sum $X + Y$ depends on how $X$ and $Y$ are related.
\begin{enumerate}
    \item $X$ and $Y$ are independent, and $P(X = 1) = P(Y = 1) = 0.5$.
    \item $P(X = 1) = 0.5$, and $P(Y = 1 | X = 1) = 1$, and $P(Y = 0 | X = 0) = 1$, so $X$ and $Y$ always match.
    \item $P(X = 1) = 0.5$, and $P(Y = 1 | X = 1) = 0$, and $P(Y = 0 | X = 0) = 0$, so $X$ and $Y$ never match.
\end{enumerate}

\item \underline{Weight Measurement}: You have a scale that you are using to weigh a very small amount of a substance that you synthesized in your lab. The scale is well-calibrated so that each time you take a measurement, the expectation of that measurement, $\mu_X$ is the true weight of the sample, but the scale is also noisy, so the standard deviation of each measurement is $\sigma_X = 0.3\textrm{ mg}$.

    You weigh the substance using this scale $n$ times, producing a set of measurements $X_1, \cdots, X_n$. Let $A_n = \frac{1}{n} \sum_{i=1}^n X_i$ be the sample mean of these measurements. You will use $A_n$ as your estimate of the true weight of the substance $\mu_X$. You want to make sure this estimate is close to the true weight $\mu_X$. 
    \begin{enumerate}
        \item What is the expectation of $A_n$, $E(A_n)$?
        \item The standard deviation of your estimate, $\mathrm{SD(A_n)}$ is a good measure of the potential error in your estimate of the weight of the substance. You would like $SD(A_n)$ to be at most $0.01\textrm{ mg}$. What is the smallest number of measurements $n$ that you would need to take for $\mathrm{SD(A_n)}$ to be this small?
    \end{enumerate}
\end{enumerate}
\end{document}
