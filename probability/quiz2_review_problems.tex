\documentclass[11pt]{article}
\usepackage{fancyhdr}
\usepackage[vmargin=3.5cm, hmargin=2cm]{geometry}
\usepackage{amsmath, amsfonts, amsthm}
\usepackage{graphicx}
\usepackage{moreverb}
\usepackage{enumerate}
\usepackage{bm}
\usepackage[tiny,compact]{titlesec}
\pagestyle{fancy}
\headheight 14pt
\lhead{UC Berkeley}
\chead{}
\rhead{Stat 88 Fall 2016}
\rfoot{}
\cfoot{\thepage}
\lfoot{}

\parindent0pt  % to stop indenting paragraphs
\parskip1.5ex  % to insert vertical space between paragraphs

% Different font in captions
\newcommand{\captionfonts}{\small}

\makeatletter  % Allow the use of @ in command names
\long\def\@makecaption#1#2{%
  \vskip\abovecaptionskip
  \sbox\@tempboxa{{\captionfonts #1: #2}}%
  \ifdim \wd\@tempboxa >\hsize
    {\captionfonts #1: #2\par}
  \else
    \hbox to\hsize{\hfil\box\@tempboxa\hfil}%
  \fi
  \vskip\belowcaptionskip}
\makeatother   % Cancel the effect of \makeatletter

\newcommand{\V}{\mathrm{Var}}
\newcommand{\E}{\mathbb{E}}
\newcommand{\mbf}{\mathbf}
\newcommand{\mr}{\mathrm}
\newcommand{\yiobs}{Y_i^\mr{obs}}
\newcommand{\yobs}{Y^\mr{obs}}
\newcommand{\yimis}{Y_i^\mr{mis}}
\newcommand{\ymis}{Y^\mr{mis}}
\newcommand{\N}{\mathcal{N}}
\newcommand{\utilde}{\underset{\widetilde{}}}
\newcommand{\txt}{\texttt}

\begin{document}
\centerline{\textbf{Quiz 2 Review Problems}}
You do not need to submit these problems. They are only to aid you in studying for the exam. Please discuss the solutions as openly and explicitly as you'd like on Piazza.

Many of these questions were taken from the Fall 2015 midterm.

For additional practice, see the problems that were not assigned for homework in chapters 4 and 5.

\begin{enumerate}
    \item A die has one red face and five blue faces. If the die is rolled 12 times, what is the chance that the red face never appears?

    \item A random number generator draws uniformly at random with replacement from the 10 digits 0, 1, 2, 3, 4, 5, 6, 7, 8, and 9. 
        \begin{enumerate}
            \item Suppose you run the generator five times. Find the chance that the same digit is drawn all five times.
            \item What is the probability that the largest of these draws is exactly 5?
            \item What is the probability that the smallest draw is no smaller than 3?
            \item What is the probability that at least one number is repeated?
        \end{enumerate}

    \item A roulette wheel has 38 distinct pockets. Each time the wheel is spun, one of the pockets is the winner. 
        \begin{enumerate}
            \item Suppose you spin the wheel 10 times and keep track of the sequence of winning pockets. How many possible sequences are there?
            \item There are exactly 2 green pockets on the wheel. What is the probability that exactly 2 of the 10 spins lands in a green pocket?
        \end{enumerate}

    \item A standard deck consists of 13 hearts, 13 diamonds, 13 clubs, and 13 spades, making 52 cards in all. 

        \begin{enumerate}
            \item Suppose cards are dealt one by one at random without replacement.
    What is the chance that the 10th card is a heart, given that the 7th card is a spade and the 32nd card is a diamond?
            \item A poker hand consists of 5 cards picked at random without replacement from the deck. 
    Find the chance that the king of hearts and the king
    of diamonds are both picked in a poker hand.
        \end{enumerate}

    \item A monkey is tapping at a keyboard that has one key for each of the 26 letters of the English alphabet. Assume that
each time the monkey is equally likely to pick any one of the 26 letters, regardless of what it has picked at other times.
        \begin{enumerate}
            \item Find the chance that the first eight letters the monkey picks can form the word KEYBOARD, by rearrangement 
        if necessary.
            \item A palindrome is a sequence of letters that reads the same forwards and backwards, like RADAR (a five-letter palindrome). 
        For our purposes, it can be a nonsense ``word'' like WSBSW or AAAAA.
        How many five-letter palindromes are there? What is the chance that the first five letters that the monkey picks, in sequential order, form
        a palindrome?
        \end{enumerate}

    \item In what follows, assume that $n$ and $N$ are positive integers. A bet is such that you have chance $1/N$ of winning it each time you play, regardless of the results of all other times. 
Suppose you play $n$ times. 

        \begin{enumerate}
            \item Find the exact chance that you win at least one of the $n$ bets, and provide an exponential approximation for the chance. Show your calculations; you do not need to prove math facts about the exponential or logarithmic functions.

            \item In terms of $N$, what is the smallest value of $n$ so that the approximate chance of winning is at least $2/3$? Prove your answer.
        \end{enumerate}

    \item You are conducting a survey to see how faculty, staff, and students in the Statistics department view the increasing popularity of data science. Suppose there are 150 people invovled with the department: 100 students, 30 faculty, and 20 staff. For your survey, you are selection 15 people, without replacement. Professors D'Amour and Adikhari are included among the faculty.
        \begin{enumerate}
            \item What is the probability that your sample includes no staff members?
            \item What is the probability that your sample includes exactly 2 faculty?
            \item What is the probability that your sample includes at least 10 students?
            \item What is the probability that your sample includes exactly 2 staff but no faculty?
            \item What is the probability that your sample includes Prof. Adikhari, but does not include Prof. D'Amour.
            \item What is the probability that your sample includes exactly one of Prof. Adhikari and Prof. D'Amour?
            \item What is the probability that your sample includes exactly one of Prof. Adhikari and Prof. D'Amour, and includes exactly 10 students?
        \end{enumerate}

    \item Suppose cars can only have one of four colors in Berkeley: red, blue, green, or white. The distribution of colors is as follows: red 10\%, blue 35\%, green 25\%, white 30\%. You stand on a street corner and watch cars 20 cars drive by. Assume that this observation process approximates sampling cars from Berkeley with replacement. (Hint: When counting cars of a specific color, it can help to think of car colors as, e.g., ``red'' or ``not red''.)
        \begin{enumerate}
            \item What is the probability that you see exactly 5 red cars?
            \item What is the probability that you see at least 10 blue cars?
        \end{enumerate}

    \item Halloween is coming soon and the children in your neighborhood will soon come trick-or-treating (that is, they will come to your door to demand candy). Suppose there are 20 children in your neighborhood and 30 houses (one of which is yours). Each child independently chooses 10 houses at random without replacement to visit.
        \begin{enumerate}
            \item What is the probability that a specific child will visit your house?
            \item What is the probability that exactly 10 children visit your house?
        \end{enumerate}

    \item (Continuing the last question.) You have bunch of candy in your house, ranging from very cheap to very fancy. Suppose that, if you ordered the candy in your house from cheapest to fanciest, the $i$th candy cost $25i$ cents to buy (so the cheapest candy cost 25 cents, the 10th cheapest cost \$2.50, etc). You want to keep the fancy candy for yourself (the kids wouldn't appreciate the fancy stuff anyway), so when a child comes to your door, you give them the cheapest remaining candy -- thus, if 10 children come to your door, you will give out the 10 cheapest pieces of candy that you have. 
        \begin{enumerate}
            \item If $k$ children come to your door, what is the total cost of the candy that you will give out? (Hint: This uses a summation identity from before the probability chapters.)
            \item What is the probability that you will give away more than \$15 worth of candy?
        \end{enumerate}
\end{enumerate}
\end{document}
